\documentclass{mcmthesis}
 %\documentclass[CTeX = true]{mcmthesis}  % 当使用 CTeX 套装时请注释上一行使用该行的设置
\mcmsetup{tstyle=\color{black}\bfseries,%修改题号,队号的颜色和加粗显示,黑色可以修改为 black
        tcn = 25M656, problem = A, %修改队号,参赛题号
        sheet = true, titleinsheet = true, keywordsinsheet = true,%修改sheet显示信息
        titlepage = false, abstract = true}


        \usepackage{setspace}%目录间距的调节
        \let\oldtableofcontents\tableofcontents
        \renewcommand{\tableofcontents}{%
          \begingroup
          \setstretch{0.1} % 设定为1.5倍行距,可修改为1.2、2等
          \oldtableofcontents
          \endgroup
        }
      %%%%%%%%%%%下面是关于references的
      \usepackage{cite} % 推荐使用,优化引用格式
      \usepackage{geometry}
      \geometry{left=1in, right=1in, top=1in, bottom=1in}
    % 调整引用为右上角标
    \makeatletter
    \def\@cite#1#2{\textsuperscript{[#1]}} % 设置右上角标格式
    \makeatother  
        
  %四款字体可以选择
  %\usepackage{times}
  \usepackage[scheme=plain]{ctex}%%%%%%%%%%%%%%%%%%%%%
  \usepackage{newtxtext,newtxmath} %CTeX 无此字体,可用 txfonts 替代,请使用新版 TeXLive.
  \usepackage[english]{babel} % 设置文档主要语言为英文%%%%%%%%%%%%%%%%%%%%
  %\usepackage{ctex}%%%%%%%%%%%%%%%%%%%%%%%%%%%%%%%%%%%%
  \usepackage{lastpage} % 获取总页数
 %\usepackage{palatino}
  %\usepackage{txfonts}
  \geometry{left=1in,right=0.75in,top=1in,bottom=1in}  % 设置1英寸的边距
\usepackage{indentfirst}  %首行缩进,注释掉,首行就不再缩进。
\usepackage{lipsum}
\usepackage{tikz}
\usetikzlibrary{shapes.geometric, arrows}
\usetikzlibrary{calc}
\graphicspath{{C:/Users/32624/Desktop/MCM/}}

\title{Neophocaena Asiaeorientalis in Vortex}
\author{\small \href{https://www.latexstudio.net/}
  {\includegraphics[width=7cm]{mcmthesis-logo}}}
\date{\today}

%%%%%%%%%%%%%%%%%%%%%%%%%%%%%%%%%%%%%%%%%(上面的部分你们不要动,有格式问题找ww)
\begin{document}
\begin{abstract}%这里是摘要部分
    \par       
\begin{keywords}%这里是关键词
               
\end{keywords}
\end{abstract}   

\maketitle
\newpage
\pagestyle{main}
\setcounter{page}{2} % 设置页码从 2 开始
\tableofcontents

\newpage
%正文
\section{Introduction}
\subsection{Problem Background}
\subsection{Restatement of the Problem}
\subsection{Literature Review}%可以不要,一般如果有半页可以要
\subsection{Our Work}
\section{Assumptions and Justifications}
\section{Notations}%可以有Definitions and Notations
\section{Model 1}
\section{Model 2}
\section{Sensitivity Analysis}
\section{Strengths and Weaknesses}%或者做成 Model Evaluation and Further Discussion多一个Model Extension小部分
\subsection{Strengths}
\subsection{Weaknesses}
\section{Conclusion}

\newpage

\renewcommand{\refname}{References} % 确保标题显示为 References
\addcontentsline{toc}{section}{References} % 手动加入目录
\patchcmd{\thebibliography}{\section*}{\section*}{}{} % 移除编号
\nocite{*}

\bibliographystyle{ieeetr} % 可选格式:plain、ieeetr、apalike 等
\bibliography{References}

\end{document}